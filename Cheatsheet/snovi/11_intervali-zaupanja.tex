\section{Intervali zaupanja}
\subsection{Interval zaupanja za neznani parameter porazdelitve}
Naj bo $\theta$ neznani parameter porazdelitve slučajne spremenljivke $X$ in $(X_1,X_2,\ldots , X_n)$ enostavni slučajni vzorec.
Iščemo interval vrednosti, v katerem se, z veliko verjetnostjo, nahaja neznani parameter $\theta$.\\
Na osnovi vzorca se definirata statistiki (funkciji vzorca) $L$ in $U$ tako da velja
\begin{equation*}
P(L\leq \theta \leq U)=1-\alpha.
\end{equation*}
Potem rečemo, da je $I_{\theta}=[L,~U]$ interval zaupanja za neznani parameter $\theta$ s \emph{stopnjo zaupanja} $1-\alpha$. Število $\alpha$ se imenuje \emph{stopnja tveganja}.

Običajno se računa $90\%$, $95\%$ ali $99\%$ interval zaupanja ($1-\alpha$ je enako $0.9$, $0.95$ ali $0.99$).	




\subsection{Interval zaupanja za pricakovano vrednost $\mu$}

\subsection{Opomba}

\subsection{Interval zaupanja za standardni odklon $\sigma$ (pricakovana vrednost $\mu$ ni znana)}

\subsection{Interval zaupanja za delez p}